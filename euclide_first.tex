\documentclass{article}
\usepackage[utf8]{inputenc}

\title{Tikz-euclid-examples}
\author{Randpunkt}
\date{\today}

\usepackage{tikz}
\usetikzlibrary{calc,intersections,through,backgrounds}
\usepackage{tkz-euclide}

\begin{document}

\maketitle

\section{Euclid’s Elements - Book I, Proposition I }
To construct an equilateral triangle on a given finite straight line.

\colorlet{input}{red!80!black}
\colorlet{output}{red!70!black}
\colorlet{triangle}{orange!40}

\begin{tikzpicture}[scale=1.5,thick,help lines/.style={thin,dotted,draw=black!50}]
    \tkzDefPoint(0,0){A}
    \tkzDefPoint(1.25+.5,0.25+.5){B}
    \tkzInterCC(A,B)(B,A) \tkzGetPoints{C}{X}
    \tkzFillPolygon[triangle,opacity=.5](A,B,X)
    \tkzDrawSegment[input](A,B)
    \tkzDrawSegments[red](A,C B,C)
    \tkzDrawCircles(A,B B,A)
    \tkzDrawPoints[fill=gray,opacity=.5](A,B,C)
    \tkzLabelPoints(A,B)
    \tkzLabelCircle[below=12pt](A,B)(180){$\mathcal{D}$}
    \tkzLabelCircle[above=12pt](B,A)(180){$\mathcal{E}$}
    \tkzLabelPoint[above,red](C){$C$}
    
\end{tikzpicture}





\end{document}
